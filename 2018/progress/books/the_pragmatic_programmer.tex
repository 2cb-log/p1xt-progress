\twocolumn
\chapter{Book: The Pragmatic Programmer}
\section{Notes}
\subsection{A Pragmatic Philosophy}
Take responsibility. Be honest and direct. Provide solutions, provide options, not excuses.    
Protect against entropy. Don't leave a mess.   
Startup fatigue cure: build minimal, then add. Remember the big picture.    
Write software that is good enough. Know when to stop.

Knowledge Portfolio - invest regularly, diversify, keep a balance between conservative and high risk/rewards, buy low sell high, review and re-balance.

\subsubsection{Goals}
\begin{itemize}
\item learn at least one new language every year
\item read a technical book each quarter
\item read nontechnical books too
\item take classes
\item stay current
\item get wired (newsgroups, online papers, etc)
\end{itemize}

\subsection{A Pragmatic Approach}

Don't repeat yourself. Make it easy to reuse. Eliminate effects between unrelated things. There are no final decisions. Use Tracer bullets to find the target. Prototype to learn. Estimate to avoid surprises. Iterate the schedule with the code.

\subsection{The Basic Tools}
Keep knowledge in plain text. Use the command line. Use a single editor well. Always use source code control. Fix the problem, not the blame. Don't Panic. Don't assume it, prove it. Write code that writes code.
\subsection{Pragmatic Paranoia}

You can't write perfect software. Design by contract. Crash early. If it can't happen, use assertions to ensure it wont. Use exceptions for exceptional problems. Finish what you start.
\subsection{Bend, or Break}

Minimize coupling between disparate portions of code. Configure, don't integrate. Put abstractions in code, details in metadata. Analyze workflow to improve concurrency. Design using services. Always design for concurrency.

\subsection{While You Are Coding}

Don't program by coincidence. Program deliberately. Estimate the efficiency of your algorithms. Refactor early, refactor often. Design to test. Test your software. Don't use Wizard code you don't understand.

\subsection{Before The Project}

Don't gather requirements, dig for them. Work with a user to think like a user. Abstractions live longer than details. Use a project glossary. Don't think outside the box, find the box. Listen to nagging doubts, start when you're ready. Some things are better done than described.

\subsection{Pragmatic Projects}

No broken windows, boil some frogs, communicate, DRY, use automation, make it good enough. Automate builds. Test early. Test often. Test automatically. Build documentation in, don't bolt it on. Gently exceed your users' expectations. Sign your work. Take responsibility for it. 

\onecolumn
